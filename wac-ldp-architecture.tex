\documentclass[10pt]{article}
\usepackage[utf8]{inputenc}

% Page setup
\usepackage[a3paper,landscape,margin=2cm]{geometry}

% Typography
\usepackage[T1]{fontenc}
\usepackage[scaled]{berasans}
\usepackage[scaled]{beramono}
\renewcommand*\familydefault{\sfdefault}
\usepackage{microtype}
\parindent 0pt

% TikZ
\usepackage{./tikz-uml}
\usetikzlibrary{positioning}

% Headings
\makeatletter
\def\@maketitle{%
  {\LARGE\bf\@title\par}
  \vskip .5em
  {\large\@author\ -- \@date\par}
  \vskip 2em
}
\makeatother

% Notes
\usepackage{multicol}
\newenvironment{Note}
  {\begin{multicols}{3}%
     \parskip 1em}
  {\end{multicols}}

% Document metadata
\title{
  wac-ldp architecture
  \it (status: draft)
}
\author{Michiel de Jong, Aaron Coburn}

\begin{document}

\maketitle


\section*{Purpose}
This document describes the code structure of inrupt's wac-ldp component.


\section*{Legend}
The architectural diagram follows standard UML notation.

For more specific symbols that are not part of UML,
Node.js/JavaScript/TypeScript conventions were used as follows:

\begin{description}
  \item[T?] represents a~value that is either not present
            or a~value of type~T.
  \item[Promise<T>] represents a~value that will asynchronously resolve
                    to a~value of type~T.
  \item[Readable<T>] represents an asynchronous one-time readable stream
                     of values of type~T.
  \item[Buffer] is an in-memory buffer of bytes,
                possibly with a~character encoding.
\end{description}


\newcommand\ResourceStoreBody{%
  + getRepresentation(ResourceIdentifier, RepresentationPreferences) : Promise<Representation>\\
  + addResource(container : ResourceIdentifier, Representation) : Promise<ResourceIdentifier>\\
  + setRepresentation(ResourceIdentifier, Representation) : Promise<void>\\
  + deleteResource(ResourceIdentifier) : Promise<void>\\
  + modifyResource(ResourceIdentifier, Patch) : Promise<void>\\
}

\section*{Diagram}
\begin{tikzpicture}

\begin{umlpackage}{Storage}
  \umlinterface[below=2.25 of Operation]{ResourceStore}{}{
    \ResourceStoreBody
  }
  \umluniaggreg[mult=1]{Operation}{ResourceStore}
\end{umlpackage}

\end{tikzpicture}

\end{document}
